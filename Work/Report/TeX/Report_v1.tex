% ----------------------------------------------------------------------------------------------------------
% Packages
% ----------------------------------------------------------------------------------------------------------

\documentclass[12pt,a4paper,bibliography=totocnumbered,listof=totocnumbered]{scrartcl}
\usepackage[english]{babel}
\usepackage[utf8]{inputenc}
\usepackage{amsmath}
\usepackage{amsfonts}
\usepackage{amssymb}
\usepackage{graphicx}
\usepackage{fancyhdr}
\usepackage{tabularx}
\usepackage{geometry}
\usepackage{setspace}
\usepackage[right]{eurosym}
\usepackage[printonlyused]{acronym}
\usepackage{subfig}
\usepackage{floatflt}
\usepackage[usenames,dvipsnames]{color}
\usepackage{colortbl}
\usepackage{paralist}
\usepackage{array}
\usepackage{titlesec}
\usepackage{parskip}
\usepackage[right]{eurosym}
%\usepackage{picins}
\usepackage[subfigure,titles]{tocloft}
\usepackage[pdfpagelabels=true]{hyperref}
\usepackage{mathdots}
\usepackage{listings}
\usepackage{lipsum}
\usepackage{booktabs}
\usepackage{fix-cm}
\usepackage{rotating}
\lstset{basicstyle=\footnotesize, captionpos=t, breaklines=true, showstringspaces=false, tabsize=2, frame=lines, numbers=left, numberstyle=\tiny, xleftmargin=2em, framexleftmargin=2em}
\makeatletter
\def\l@lstlisting#1#2{\@dottedtocline{1}{0em}{1em}{\hspace{1,5em} Lst. #1}{#2}}
\makeatother

\usepackage{pdflscape}
%\usepackage{lscape}

\geometry{a4paper, top=27mm, left=35mm, right=15mm, bottom=35mm, headsep=10mm, footskip=12mm}

\hypersetup{unicode=false, pdftoolbar=true, pdfmenubar=true, pdffitwindow=false, pdfstartview={FitH},
	pdftitle={Bachelor Thesis},
	pdfauthor={Felix Gutmann},
	pdfsubject={Bachelor Thesis},
	pdfcreator={\LaTeX\ with package \flqq hyperref\frqq},
	pdfproducer={pdfTeX \the\pdftexversion.\pdftexrevision},
	pdfkeywords={Bachelor Thesis},
	pdfnewwindow=true,
	colorlinks=true,linkcolor=black,citecolor=black,filecolor=magenta,urlcolor=black}
\pdfinfo{/CreationDate (D:20110620133321)}
\DeclareMathOperator*{\argmin}{arg\,min}
\begin{document}

\titlespacing{\section}{0pt}{12pt plus 4pt minus 2pt}{-6pt plus 2pt minus 2pt}

% Headers and footers

\renewcommand{\sectionmark}[1]{\markright{#1}}
\renewcommand{\leftmark}{\rightmark}
\pagestyle{fancy}
\lhead{}
\chead{}
\rhead{\thesection\space\contentsname}
%\lfoot{Complex Economic Systems - An analytical approach to Input-Output tables\newline}
\cfoot{}
\rfoot{\ \linebreak \thepage}
\renewcommand{\headrulewidth}{0.4pt}
\renewcommand{\footrulewidth}{0.4pt}

% ----------------------------------------------------------------------------------------------------------
%Prefix
% ----------------------------------------------------------------------------------------------------------

\renewcommand{\thesection}{\Roman{section}}
\renewcommand{\theHsection}{\Roman{section}}
\pagenumbering{Roman}

% ----------------------------------------------------------------------------------------------------------
% Title
% ----------------------------------------------------------------------------------------------------------

\thispagestyle{empty}
\begin{center}
	\includegraphics[width=\textwidth]{Pictures/logo01.jpg}\\
	\vspace*{2cm}
	\vspace*{2cm}
	\Huge
	\textbf{Master Thesis}\\
	\vspace*{0.5cm}
	\large
	\textbf{Topic:}\\
	\vspace*{1cm}
	\textbf{Unsupervised learning in decision making}\\
	\vspace*{2cm}
\end{center}	

$\vspace{6cm}$
\begin{tabbing}
\hspace*{1cm}\=\hspace*{3.2cm}\=\hspace*{3cm}\=\hspace*{2.7cm}\= \kill
\onehalfspacing
\textbf{Author:} \>\> Domagoj Fizulic\\
\textbf{} \>\> Felix Gutmann\\
\textbf{Student number:} 	\>\> 125604\\
\textbf{Program:} \>\> M.S. Data Science\\
\textbf{E-Mail:} \>\> domagoj.fizulic@barcelonagse.eu\\
\textbf{} \>\> felix.gutmann@barcelonagse.eu
\end{tabbing}
\vspace{1cm}
\pagebreak

% ----------------------------------------------------------------------------------------------------------
% Declaration
% ----------------------------------------------------------------------------------------------------------



\pagebreak

% ----------------------------------------------------------------------------------------------------------
% Abstract
% ----------------------------------------------------------------------------------------------------------


\onehalfspacing

\titlespacing{\section}{0pt}{12pt plus 4pt minus 2pt}{2pt plus 2pt minus 2pt}



\pagebreak

% ----------------------------------------------------------------------------------------------------------
% Index
% ----------------------------------------------------------------------------------------------------------

% TODO Typ vor Nummer

\renewcommand{\cfttabpresnum}{Tab. }
\renewcommand{\cftfigpresnum}{Fig. }
\settowidth{\cfttabnumwidth}{Fig. 10\quad}
\settowidth{\cftfignumwidth}{Fig. 10\quad}

\titlespacing{\section}{0pt}{12pt plus 4pt minus 2pt}{2pt plus 2pt minus 2pt}
\singlespacing
\rhead{Table of contents}
\renewcommand{\contentsname}{I Table of Contents}
\phantomsection
\addcontentsline{toc}{section}{\texorpdfstring{I \hspace{0.35em}Table of Contents}{Table of Contents}}
\addtocounter{section}{1}


% ----------------------------------------------------------------------------------------------------------
% Table of contents
% ----------------------------------------------------------------------------------------------------------
\setcounter{page}{1}

\rhead{Table of Contents}

	\tableofcontents

\pagebreak

% ----------------------------------------------------------------------------------------------------------
% List of figures
% ----------------------------------------------------------------------------------------------------------

\rhead{List of Figures}

	\listoffigures
	
\pagebreak

% ----------------------------------------------------------------------------------------------------------
% List of tables
% ----------------------------------------------------------------------------------------------------------

\rhead{List of Tables}

	\listoftables
	
\pagebreak

%----------------------------------------------------------------------------------------------------------
% List of Listings
% ----------------------------------------------------------------------------------------------------------

\rhead{List of Listings}
\renewcommand{\lstlistlistingname}{List of Listings}
{\labelsep2cm\lstlistoflistings}
\pagebreak

%----------------------------------------------------------------------------------------------------------
% List of Symbols
% ---------------------------------------------------------------------------------------------------------

\renewcommand{\arraystretch}{1.5}	
\section{List of mathematical symbols}
\rhead{List of mathematical Symbols}

\begin{tabular}{p{6cm}p{9cm}}
\textbf{Symbol} 		& 		\textbf{Meaning} \\
\midrule
\vspace{0.3cm} & \vspace{0.3cm} 			\\
$a_t$				  & Action at time t			\\
$Q(a)_t$	& Value function at time t \\
$\epsilon$			& Probability of exploration in epsilon greedy \\
$\alpha$ & Learning rate \\ 
$\tau$ & Softmax parameter \\
\end{tabular}

\pagebreak

%----------------------------------------------------------------------------------------------------------
% List of abbreviations
% ---------------------------------------------------------------------------------------------------------

\section{List of abbreviations}
\rhead{List of Abbreviations}

\begin{tabular}{p{6cm}p{9cm}}
\textbf{Abbreviations} & \textbf{Description} 										\\
\midrule
\vspace{0.3cm} & \vspace{0.3cm} 														\\ 
\end{tabular}

\newpage

% ----------------------------------------------------------------------------------------------------------
% Prefix 2
% ----------------------------------------------------------------------------------------------------------

% Title spacing

\titlespacing{\section}{0pt}{12pt plus 4pt minus 2pt}{-6pt plus 2pt minus 2pt}
\titlespacing{\subsection}{0pt}{12pt plus 4pt minus 2pt}{-6pt plus 2pt minus 2pt}
\titlespacing{\subsubsection}{0pt}{12pt plus 4pt minus 2pt}{-6pt plus 2pt minus 2pt}

% Header

\renewcommand{\sectionmark}[1]{\markright{#1}}
\renewcommand{\subsectionmark}[1]{}
\renewcommand{\subsubsectionmark}[1]{}
\lhead{Chapter \thesection}
\rhead{\rightmark}

\onehalfspacing

\renewcommand{\thesection}{\arabic{section}}
\renewcommand{\theHsection}{\arabic{section}}
\setcounter{section}{0}
\pagenumbering{arabic}
\setcounter{page}{1}

%RGB Colour set

\definecolor{persblue}{rgb}{0.0862745,0.211765,0.360784}
\definecolor{persred}{rgb}{0.388235,0.145098,0.137255}
\definecolor{persgray}{rgb}{0.501961,0.501961,0.501961}
\definecolor{persgreen}{rgb}{0.054902,0.411765,0.352941}

%---------------------------------------------------------------------------------------------------------
% 1. Introduction
%---------------------------------------------------------------------------------------------------------

\section{Introduction and conceptual approach}


\section{Simmulation}

\subsection{Data generation}

Challange: 
Categorical data with time series attributes. 
Apply transformation choice probabilities ()
Define similarity meassures on categories

\subsection{Reinforcement Learning backround}

\textit{Reinforcement Learning} (RIL) is a branch of \textit{Machine Learning} try model how an artificial agents interact with its environment and learns from the process over time. \\
In particular an agents is confronted with the task of choosing sequently from a set of choices. In comparison to \textit{supervised learning}, where an agent is learning based on set of examples an agent in RIL  The objective is to let the agent learn within a certain the optimal action. Since it doesn't have any apriori information about the system it has to explore new possible action and so has to deviate from the optimal action. Furthermore, it has to keep track of value of each action he did so far. So the main task of the agent is to balance exploration and explotation. There are to basic approaches to model this trade-off; An \textit{"Epsilon-Greedy"} selection method and Soft \textit{"Softmax"} selection method.\\
Considering first epsilon greedy action selection mehtod: Let $Q_t(a)$ be the value function of action. In general select the next action as $a$ as $a_{t+1} = \arg \max Q_t(a)$. However, to model exploration there is a probability to deviate from that greedy strategy. Choose next action Let $p_e$ be the probability of exploration. $p=\epsilon$

$Q_t(a) = \frac{R_1 + R_2 + \dots + R_{K_\alpha}}{K_\alpha} $

$Q_{k+1} = Q_k + \alpha\left[ R_k -  Q_k	 \right]$ 

There are two basic 

 $a$. \textit{"Epsilon Greedy"}



Soft Max Selection 

$P(a_{t}|X) = \frac{e^{\frac{Q_t(a)}{\tau}}}{\sum_{i}^{K} e^{\frac{Q_t(i)}{\tau}}}$


\subsection{Unsupervised Learning}


\subsection{Simmulation results}

\pagebreak
% ----------------------------------------------------------------------------------------------------------
% Literature
% ----------------------------------------------------------------------------------------------------------

\renewcommand\refname{List of Literature}

\bibliographystyle{apalike}

\bibliography{bib}


\pagebreak

%----------------------------------------------------------------------------------------------------------
% Appendix
% ---------------------------------------------------------------------------------------------------------
\lhead{Appendix \thesection}
\rhead{}
\pagenumbering{Alph}
\setcounter{page}{1}

\begin{appendix}
	
	\section*{Appendix}
	\phantomsection
	\addcontentsline{toc}{section}{Appendix}
	\addtocontents{toc}{\vspace{-0.5em}}

\end{appendix}

\pagebreak
%----------------------------------------------------------------------------------------------------------
% End Document
%----------------------------------------------------------------------------------------------------------


\begin{sidewaystable}[!h]
	\centering
	\begin{tabular}{| l || c |cl | c | c | c | c | c | c || c  |  c | c | c | c | c | }
		\toprule \toprule
		\textbf{Specification} &$\boldsymbol{\mu}$ & $\boldsymbol{\sigma}$ & \textbf{Cluster Size} & \textbf{Seed} & \textbf{Decision Function} & $\boldsymbol{\alpha}$  &  $\boldsymbol{\tau}$  & \textbf{N} & \textbf{MI} & \textbf{NMI} &  \textbf{AMI} &  \textbf{CS} &  \textbf{HS } & \textbf{VMS}     \\
		\hline
		1 & -  & -& -& -& -& -& -& -& -& -& -& -& - & - \\
		\bottomrule
	\end{tabular}
	\caption{Overview Outcome }
\end{sidewaystable}


\end{document}